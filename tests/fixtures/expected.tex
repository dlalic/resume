\documentclass[10pt]{article}
\usepackage[margin=1.2cm]{geometry}
\geometry{a4paper}
\usepackage{graphicx}
\usepackage[space]{grffile}
\usepackage{fontspec}
\usepackage{enumitem}
\usepackage{paracol}
\usepackage{tikz}
\usepackage{wrapfig}
\usepackage{xltxtra}
\usepackage{xunicode}
\defaultfontfeatures{Mapping=tex-text}
\setmainfont{FiraSans}[
    Path = ./examples/fonts/,
    UprightFont = *-Light,
    BoldFont = *-Bold,
    ItalicFont = *-Italic,
]
\setcounter{secnumdepth}{0}
\setlist[itemize]{before=\small,leftmargin=*,rightmargin=0.5em,noitemsep}

\begin{document}


  
    \section{Jane Doe}
    Berlin, Germany\\
    foo@bar.com


\columnratio{0.7}

\begin{paracol}{2}
\begin{leftcolumn}
\section{Employment}

\begin{description}
    \item [Acme Corp] Code monkey, Jan 2017 - Sep 2019
    \item [\footnotesize{
     foo,  bar,  baz 
    }]
\end{description}
\begin{itemize}
    
    \item Lorem Ipsum is simply dummy text of the printing and typesetting industry.
    
    \item Lorem Ipsum has been the industry's standard dummy text ever since the 1500s, when an unknown printer took a galley of type and scrambled it to make a type specimen book.
    
\end{itemize}

\section{Education}

\begin{description}
    \item [Mars Univeristy]
\end{description}
\begin{itemize}
    \item dfg.gf.gdf.dfg.gdf.dgf, 2017 - 2017
\end{itemize}

\end{leftcolumn}

\begin{rightcolumn}
\section{Skills}

    \begin{description}
    \item [Web:]  Rails,  Django 
    \end{description}
	\definecolor{web_color_start}{RGB}{36, 36, 93}
	\definecolor{web_color_end}{RGB}{115, 199, 238}
    \noindent\begin{tikzpicture}
	\foreach \x in {1,...,3}
	{ \pgfmathtruncatemacro{\percent}{100 - \x /4 * 100}
	  \node[rectangle,draw=none,fill=web_color_start!\percent!web_color_end,minimum width=30] (0,0) at (\x, 0) { };
	}
	\end{tikzpicture}

    \begin{description}
    \item [Mobile:]  Android 
    \end{description}
	\definecolor{mobile_color_start}{RGB}{175, 48, 51}
	\definecolor{mobile_color_end}{RGB}{240, 177, 86}
    \noindent\begin{tikzpicture}
	\foreach \x in {1,...,4}
	{ \pgfmathtruncatemacro{\percent}{100 - \x /5 * 100}
	  \node[rectangle,draw=none,fill=mobile_color_start!\percent!mobile_color_end,minimum width=30] (0,0) at (\x, 0) { };
	}
	\end{tikzpicture}

\section{Experience}

\definecolor{maintenance_color}{RGB}{218, 64, 122}

\definecolor{greenfield_color}{RGB}{95, 175, 87}

\def\angle{0}
\def\radius{3}
\def\cyclelist{{"maintenance_color","greenfield_color",
}}
\newcount\cyclecount \cyclecount=-1
\newcount\ind \ind=-1
\begin{tikzpicture}[scale=0.9]
    \foreach \percent/\name in {60/Maintenance,40/Greenfield,
    } {
    \ifx\percent\empty\else
    \global\advance\cyclecount by 1
    \global\advance\ind by 1
    \ifnum3<\cyclecount
    \global\cyclecount=0
    \global\ind=0
    \fi
    \pgfmathparse{\cyclelist[\the\ind]}
    \edef\color{\pgfmathresult}
    \draw[fill={\color},draw={\color}] (0,0) -- (\angle:\radius)
    arc (\angle:\angle+\percent*3.6:\radius) -- cycle;
    \node[text=white] at (\angle+0.5*\percent*3.6:0.65*\radius) {\scriptsize\name};
    \pgfmathparse{\angle+\percent*3.6}
    \xdef\angle{\pgfmathresult}
    \fi
    };
\end{tikzpicture}
\end{rightcolumn}
\end{paracol}

\end{document}